\documentclass{article}
\usepackage{amsmath, nccmath}  % For matrices and math
\usepackage[utf8]{inputenc}
\usepackage{xcolor}
\usepackage{colortbl}
\usepackage{amssymb}

\begin{document}

\title{Numerical Analysis}
\author{Ahmed Mohamed}
\date{\today}


\maketitle

\newpage

\section{Introduction}

\paragraph{Matrix:} Rectangular array of numbers, symbols or expressions arranged in rows and column

\bigskip

\noindent\textbf{Ex:} $2x3$ matrix would look like this

\[
A = \begin{bmatrix}
  a_{11} & a_{12} & a_{13} \\
  a_{21} & a_{22} & a_{23} 
\end{bmatrix}
\]

The number of elements in the matrix is $2 \times 3= 6$

\paragraph{Remark:} Matrices are used in different mathematical operations such as addition, multiplication, and finding determinants and inverses.

\bigskip

\noindent\textbf{General form for the matrix $m \times n$}
\[
A = \begin{bmatrix}
  a_{11} & a_{12} & a_{13} & \dots & a_{1n} \\
  a_{21} & a_{22} & a_{23} & \dots & a_{2n} \\
  a_{31} & a_{32} & a_{33} & \dots & a_{3n} \\
  \vdots & \vdots & \vdots & \ddots & \vdots \\
  a_{m1} & a_{m2} & a_{m3}& \dots & a_{mn}
\end{bmatrix}
\]


\noindent\textbf{Matrix Applications:}
\begin{enumerate}
  \item Computer Graphics
    \begin{itemize}
      \item Scaling and rotation images
      \item 3D
    \end{itemize}
  \item Machine Learning: 
      \begin{itemize}
      \item Features of data points in algorithms, Such as linear regression, Neural networks
    \end{itemize}
    \item Economics
    \begin{itemize}
      \item Building an economy system
      \item Analyze the strategies
    \end{itemize}
    
    \item Single processing
      \begin{itemize}
        \item Can be used in Image processing
      \end{itemize}
    \item Control systems
      \begin{itemize}
        \item Finding the relationship between inputs/outputs in databases
      \end{itemize}
    \item Statistics
      \begin{itemize}
        \item Regression analysis
      \end{itemize}
\end{enumerate}

\newpage

\section{Matrices Types}

\begin{enumerate}
  \item \textbf{Row Matrix:} Have only one row\par
  \textbf{EX: } $A = 1 \times 3$\par
  \[
    A = \begin{bmatrix}
      4 & 0 & \frac{1}{3}
    \end{bmatrix}
  \]

  \item \textbf{Column Matrix:} Have only one column\par
    \textbf{EX: } $A = 3 \times 1$\par
    \[
      A = \begin{bmatrix}
        5 \\ 
        33 \\ 
        0.4 \\
      \end{bmatrix}
    \]
  
  \item \textbf{Square Matrix:} Have an equal number of rows and columns\par
    \textbf{EX: } $A = 3 \times 3$\par
    \[
      A = \begin{bmatrix}
        5 & -1 & 2   \\ 
        33 & 3 & 20  \\ 
        0.4 & -3 & 0 \\
      \end{bmatrix}
    \]
    \textbf{EX: } $B = 1 \times 1$\par
    \[
      B = \begin{bmatrix}
        3.14
      \end{bmatrix}
    \]
  
  \item \textbf{Diagonal Matrix:} All non-diagonal elements are zeros\par
    \textbf{EX: } $A = 3 \times 3$\par
    \[
      A = \begin{bmatrix}
        \fbox{3} & 0 & 0 \\ 
        0 & \fbox{3} & 0  \\ 
        0 & 0 & \fbox{8} \\
      \end{bmatrix}
    \]
  \item \textbf{Scalar Matrix:} Square matrix with all non-diagonal elements are zero and all diagonal elements are equal\par
    \textbf{EX: } $A = 3 \times 3$\par
    \[
      A = \begin{bmatrix}
        \fbox{3} & 0 & 0 \\ 
        0 & \fbox{3} & 0  \\ 
        0 & 0 & \fbox{3} \\
      \end{bmatrix}
    \]

  \item \textbf{Unit Matrix(Identity):} Type of diagonal matrix, its all diagonal elements equal to $1$\par
    \textbf{EX: } $A = 3 \times 3$\par
    \[
      A = \begin{bmatrix}
        \fbox{1} & 0 & 0 \\ 
        0 & \fbox{1} & 0  \\ 
        0 & 0 & \fbox{1} \\
      \end{bmatrix}
    \]

  \item \textbf{Null(Zero) Matrix:} 
  All elements equal to $0$\par
  \textbf{EX: } $A = 3 \times 3$\par
    \[
      A = \begin{bmatrix}
        0 & 0 & 0 \\ 
        0 & 0 & 0  \\ 
        0 & 0 & 0 \\
      \end{bmatrix}
    \]

  \item \textbf{Symmetric Matrix:} 
  If the matrix $A$ is equal to $A^\top$\\Where \\
  \textbf{$A^\top$}: is the transpose matrix from $A$, Generated by switching rows and columns \par
  \textbf{EX: } Check if $A = A^\top$\par
  \[
    A = \begin{bmatrix}
      -1 & 2 & 10 \\ 
      30 & 0 & 1  \\ 
      -3 & -2 & -5 \\
    \end{bmatrix}
    \rightarrow
    A^\top = \begin{bmatrix}
      -1 & 30 & -3 \\
      2 & 0 & -2 \\
      10 & 1 & -5 \\
    \end{bmatrix}
  \]
  $\therefore A \neq A^\top$ \\
  $\therefore A$ Is not a symmetric matrix
\end{enumerate}

\section{Addition And Subtraction}

Two matrices can be added or subtracted if and only if the have the same order which is mean the same numbers of rows and columns\par
\bigskip
\textbf{Ex: } Find $A+B$ and $A-B$ Where
\[
  A = \begin{bmatrix}
    2 & 0 \\
    -5 & 6 \\
  \end{bmatrix}
  B = \begin{bmatrix}
    3 & 6 \\
    4 & 1 \\
  \end{bmatrix}
\]
\textbf{Sol:}
\[
  A+B = \begin{bmatrix}
    2 + 3 & 0 + 6 \\
    -5 + 4 & 6 + 1
  \end{bmatrix}
  \rightarrow
  A+B = \begin{bmatrix}
  5 & 6 \\
  -1 & 7
  \end{bmatrix}
\]

\[
  A-B = \begin{bmatrix}
    2 - 3 & 0 - 6 \\
    -5 - 4 & 6 - 1
  \end{bmatrix}
  \rightarrow
  A-B = \begin{bmatrix}
  -1 & -6 \\
  -9 & 5
  \end{bmatrix}
\]

\bigskip

\textbf{Ex: } Find $A+B$ Where
\[
  A = \begin{bmatrix}
    1 & 5 \\
    6 & 7 \\
    8 & 9 \\
  \end{bmatrix}
  B = \begin{bmatrix}
    3 & 6 \\
    4 & 1 \\
  \end{bmatrix}
\]
\textbf{Sol: } $A+B$ Is not defined since they have different orders \par

\newpage

\noindent\textbf{Properties Of Addition}:
\begin{enumerate}
  \item $A+B = B+A$
  \item $(A + B) + C = A + (B + C)$
  \item $A + 0 \;(\text{zero matrix}) = A$
  \item $A-B \neq B-A$
  \item $(A - B) - C = A - (B - C)$
  \item $A^{\top^\top} = A$
  \item $aA^\top = aA^\top$ Where $a$ is a number
  \item $(A+B)^\top = B^\top+A^\top$
\end{enumerate}
\bigskip
\textbf{H.W(1): }Prove that all the properties of matrix addition are satisfied 
\[
  A = \begin{bmatrix}
    -1 & 3 & -2 \\
    0 & 1 & -10 \\
    3 & 5 & 7 \\
  \end{bmatrix}
  \quad
  B = \begin{bmatrix}
  6 & -5 & -3 \\
  -2 & 1 & -0 \\
  4 & -7 & 10 \\
  \end{bmatrix}
  \quad
  C = \begin{bmatrix}
    8 & 6 & -1 \\
    20 & 0 & 10 \\
    -10 & 3 & -30 \\
  \end{bmatrix}
\]

\section{Matrix multiplication}
To defined the multiplication of two matrices $AB$, If and only if the number of columns in $A$ is equal to the number of rows in $B$

\textbf{Ex: } Find $AB$ Where
\[
  A = \begin{bmatrix}
    1 & 2 \\
    3 & 4 \\
    0 & 1
  \end{bmatrix}
  \quad
  B = \begin{bmatrix}
    2 & 1  & 4 \\
    3 & 0 & 5
  \end{bmatrix}
\]
\textbf{Sol: }
\[
  AB = \begin{bmatrix}
    1 \times 2 + 2 \times 3 & 1 \times 1 + 2 \times 0 & 1 \times 4 + 2 \times 5 \\
    3 \times 2 + 4 \times 3 & 3 \times 1 + 4 \times 0 & 3 \times 4 + 4 \times 5 \\
    0 \times 2 + 1 \times 3 & 0 \times 1 + 1 \times 0 & 0 \times 4 + 1 \times 5 \\
  \end{bmatrix}
  \Rightarrow
  AB = \begin{bmatrix}
    8 & 1 & 14 \\
    18 & 3 & 32 \\
    3 & 0 & 5 \\
  \end{bmatrix}
\]
\textbf{Properties of multiplication}:
\begin{enumerate}
  \item $A(B + C) = AB + AC$
  \item $(B + C)A = BA + CA$
  \item $AB \neq BA$
  \item $AB(C) = A(BC) = ABC$
  \item $AI = IA = A$
  \item $AA^{-1} = I$ \\
  Where $A^{-1}$: is the inverse of $A$
  \item $(AB)^\top = B^\top A^\top $
\end{enumerate}

\newpage

\section{Determinant of matrix} 
Single number can be found only for $2\times2$ matrices, Will be denoted by $det(A)$ or $|A|$
\[
A = \begin{bmatrix}
  a_{11} & a_{12} \\
  a_{21} & a_{22} \\
\end{bmatrix}
\]
\[
det(A) = |A| = a_{11} \times a_{22} - a_{12} \times a_{21}
\]

\bigskip

\textbf{Ex: } Find Determinant of $A$ Where
\[
A = \begin{bmatrix}
  6 & 4 \\ 
  7 & 9
\end{bmatrix}
\Rightarrow
|A| = 6 \times 9 - 4 \times 7 
\Rightarrow
|A| = 26
\]

For a $3\times3$ matrix
\[
A = \begin{bmatrix}
  a_{12} & a_{12} & a_{13} \\
  a_{21} & a_{22} & a_{23} \\
  a_{31} & a_{32} & a_{33} \\
\end{bmatrix}
\]

\bigskip

\[
A = a_{11} \begin{vmatrix}
  a_{22} & a_{23} \\
  a_{32} & a_{33} \\
  \end{vmatrix}
  -
  a_{12}\begin{vmatrix}
    a_{21} & a_{23} \\
    a_{31} & a_{33} \\
  \end{vmatrix}
  +
  a_{13} \begin{vmatrix}
    a_{21} & a_{22} \\
    a_{31} & a_{32} \\
  \end{vmatrix}
\]
\[
  |A| = a_{11} (a_{22} a_{33} - a_{23} a_{32})
  - a_{12} (a_{21} a_{33} - a_{23} a_{31})
  + a_{13} (a_{21} a_{32} - a_{22} a_{31})
\]

\textbf{Ex: }Find $|A|$ \\
\indent Where
\[
  A = \begin{bmatrix}
    1 & 2 & 4 \\
    0 & -1 & 0 \\
    -2 & 0 & 3
  \end{bmatrix}
\]
\indent\textbf{Sol: }
\[
  A = 1 \times \begin{bmatrix}
    -1 & 0 \\
    0 & 3 
  \end{bmatrix}
  - 
  2 \times \begin{bmatrix}
    0 & 0 \\
    -2 & 3 \\
  \end{bmatrix}
  + 4 \times \begin{bmatrix}
    0 & -1 \\
    -2 & 0 \\
  \end{bmatrix}
\]
\bigskip
\[
|A| = 1(-1 \times 3 - 0 \times 0 ) - 2(0 \times 3 - 0 \times -2) + 4(0 \times 0 - 1 \times -2) \rightarrow |A| = 5
\]

\bigskip

\newpage

\section{Cofactor of a matrix}

\[
let \; A = \begin{bmatrix}
  a_{11} & a_{12} & a_{13} \\
  a_{21} & a_{22} & a_{23} \\
  a_{31} & a_{32} & a_{33} \\
\end{bmatrix}
\]
We can find the Cofactor of $A$ by

\[ C_{ij} = (-1)^{i+j} A_{ij} \]

\[
C_{11} = -1^{2} \begin{vmatrix}
  a_{22} & a_{23} \\
  a_{32} & a_{33} \\
\end{vmatrix}
\Rightarrow C_{11} = a_{22} a_{33} - a_{23} a_{32}
\]

\[
C_{12} = (-1)^{1+2}
\begin{vmatrix}
a_{21} & a_{23} \\
a_{31} & a_{33}
\end{vmatrix}
\Rightarrow
C_{12} = -\big(a_{21}a_{33}-a_{23}a_{31}\big)\]

\[
C_{13} = (-1)^{1+3}
\begin{vmatrix}
a_{21} & a_{22} \\
a_{31} & a_{32}
\end{vmatrix}
\Rightarrow
C_{13} = a_{21}a_{32}-a_{22}a_{31}
\]

\[
C_{21} = (-1)^{2+1}
\begin{vmatrix}
a_{12} & a_{13} \\
a_{32} & a_{33}
\end{vmatrix}
\Rightarrow
C_{21} = -\big(a_{12}a_{33}-a_{13}a_{32}\big)
\]

\[
C_{22} = (-1)^{2+2}
\begin{vmatrix}
a_{11} & a_{13} \\
a_{31} & a_{33}
\end{vmatrix}
\Rightarrow
C_{22} = a_{11}a_{33}-a_{13}a_{31}
\]

\[
C_{23} = (-1)^{2+3}
\begin{vmatrix}
a_{11} & a_{12} \\
a_{31} & a_{32}
\end{vmatrix}
\Rightarrow
C_{23} = -\big(a_{11}a_{32}-a_{12}a_{31}\big)
\]

\[
C_{31} = (-1)^{3+1}
\begin{vmatrix}
a_{12} & a_{13} \\
a_{22} & a_{23}
\end{vmatrix}
\Rightarrow
C_{31} = a_{12}a_{23}-a_{13}a_{22}
\]

\[
C_{32} = (-1)^{3+2}
\begin{vmatrix}
a_{11} & a_{13} \\
a_{21} & a_{23}
\end{vmatrix}
\Rightarrow
C_{32} = -\big(a_{11}a_{23}-a_{13}a_{21}\big)
\]

\[
C_{33} = (-1)^{3+3}
\begin{vmatrix}
a_{11} & a_{12} \\
a_{21} & a_{22}
\end{vmatrix}
\Rightarrow
C_{33} = a_{11}a_{22}-a_{12}a_{21}
\]

\textbf{Ex(2): } Find the cofactor for the following
\[
A = \begin{bmatrix}
  3 & 4 & -1 \\
  1 & 0 & 3 \\
  2 & 5 & -4
\end{bmatrix}
\]

\[
C_{11} = (-1)^{1+1}
\begin{vmatrix}
0 & 3 \\
5 & -4
\end{vmatrix}
\Rightarrow
C_{11} = 0\cdot(-4)-3\cdot5 = -15
\]

\[
C_{12} = (-1)^{1+2}
\begin{vmatrix}
1 & 3 \\
2 & -4
\end{vmatrix}
\Rightarrow
C_{12} = -\big(1\cdot(-4)-3\cdot2\big) = -(-4-6) = 10
\]

\[
C_{13} = (-1)^{1+3}
\begin{vmatrix}
1 & 0 \\
2 & 5
\end{vmatrix}
\Rightarrow
C_{13} = 1\cdot5-0\cdot2 = 5
\]


\[
C_{21} = (-1)^{2+1}
\begin{vmatrix}
4 & -1 \\
5 & -4
\end{vmatrix}
\Rightarrow
C_{21} = -\big(4\cdot(-4)-(-1)\cdot5\big) = -(-16+5) = 11
\]

\[
C_{22} = (-1)^{2+2}
\begin{vmatrix}
3 & -1 \\
2 & -4
\end{vmatrix}
\Rightarrow
C_{22} = 3\cdot(-4)-(-1)\cdot2 = -12+2 = -10
\]

\[
C_{23} = (-1)^{2+3}
\begin{vmatrix}
3 & 4 \\
2 & 5
\end{vmatrix}
\Rightarrow
C_{23} = -\big(3\cdot5-4\cdot2\big) = -(15-8) = -7
\]

\[
C_{31} = (-1)^{3+1}
\begin{vmatrix}
4 & -1 \\
0 & 3
\end{vmatrix}
\Rightarrow
C_{31} = 4\cdot3-(-1)\cdot0 = 12
\]

\[
C_{32} = (-1)^{3+2}
\begin{vmatrix}
3 & -1 \\
1 & 3
\end{vmatrix}
\Rightarrow
C_{32} = -\big(3\cdot3-(-1)\cdot1\big) = -(9+1) = -10
\]

\[
C_{33} = (-1)^{3+3}
\begin{vmatrix}
3 & 4 \\
1 & 0
\end{vmatrix}
\Rightarrow
C_{33} = 3\cdot0-4\cdot1 = -4
\]

\[
C = \begin{bmatrix}
  -15 & 10 & 5 \\
  11 & -10 & -7 \\
  12 & -10 & -4 \\
\end{bmatrix}
\]

\bigskip
\bigskip
\bigskip

\textbf{Remark: } if $A$ is a matrix of $2\times2$ you can find the cofactor like this

\[
A = \begin{bmatrix}
  a_{11} & a_{12} \\
  a_{21} & a_{22}
\end{bmatrix}
\Rightarrow
C = \begin{bmatrix}
  a_{22} & -a_{21} \\
  -a_{12} & a_{11}
\end{bmatrix}
\]


\textbf{Ex: } Find cofactor for $A$
\indent Where
\[
A = \begin{bmatrix}
  -2 & -4 \\
  10 & 5 \\
\end{bmatrix}
\]

\indent \textbf{Sol: }

\[
A = \begin{vmatrix}
  -2 & -4 \\
  10 & 5 \\
\end{vmatrix}
\quad
\Rightarrow
\quad
C = \begin{vmatrix}
  5 & -10 \\
  4 & -2
\end{vmatrix}
\]

\newpage
\section{Adjoint Matrix}

Adj$(A)$ Can be found 
$adj(A) = C^\top$ Where $C$ is the cofactor of $A$

\bigskip

\textbf{Ex(3): } Find adjoint for $A$ from the example number 2

\[
C = \begin{bmatrix}
  15 & 10 & 5 \\
  11 & -10 & -7 \\
  12 & -10 & -4 \\
\end{bmatrix}
\]

\textbf{Sol: }

\[
adj(A) = C^\top = \begin{bmatrix}
  -15 & 11 & 12 \\
  10 & -10 & -10 \\
  5 & -7 & -4 \\
\end{bmatrix}
\]

\bigskip

\section{Inverse Matrix $(A^{-1})$}
You can find the inverse from this form
$A^{-1} = \frac{adj(A)}{|A|}$, Where $|A| \neq 0$

\bigskip
If $|A| = 0$ Then $A$ doesn't have inverse
\bigskip


\textbf{Ex: } Find $A^{-1}$ from Ex(3)

\[
adj(A) = C^\top = \begin{bmatrix}
  -15 & 11 & 12 \\
  10 & -10 & -10 \\
  5 & -7 & -4 \\
\end{bmatrix}
\]

\textbf{Sol: }

\bigskip

After Finding the cofactor it will equal $|A| = -10$

\[
A^{-1} = \frac{adj(A)}{|A|} = \begin{bmatrix}
  \frac{15}{10} & -\frac{11}{10} & -\frac{12}{10} \\ 
  \\
  -\frac{10}{10} & \frac{10}{10} & \frac{10}{10} \\
  \\
  -\frac{5}{10} & \frac{7}{10} & \frac{4}{10}
\end{bmatrix}
\Rightarrow
A^{-1} = \begin{bmatrix}
  1.5 & -1.1 & -1.2 \\
  -1 & 1 & 1 \\
  -0.5 & 0.7 & 0.4 \\
\end{bmatrix}
\]

\newpage
\section{Lower \& Upper Triangular Matrix}
\paragraph{Lower Triangular Matrix} All the elements above the main diagonal are zeros\par

\bigskip

\textbf{Ex: } $3 \times 3$ Lower Triangular matrix

\[
A = \begin{bmatrix}
  a_{11} & \fbox{0} & \fbox{0} \\
  a_{21} & a_{22} & \fbox{0} \\
  a_{31} & a_{32} & a_{33}
\end{bmatrix}
\]

\paragraph{Upper Triangular Matrix}All the elements below the main diagonal are zeros\par
\bigskip
\textbf{Ex: } $3 \times 3$ Upper Triangular matrix

\[
A = \begin{bmatrix}
  a_{11} & a_{12} & a_{13} \\
  \fbox{0} & a_{22} & a_{23} \\
  \fbox{0}  & \fbox{0}  & a_{33}
\end{bmatrix}
\]

\textbf{Ex: } Transfer $A$ to upper triangular matrix, Where
\[
A = \begin{bmatrix}
  0 & 4 & 2 \\ 
  2 & 3 & 5 \\
  3 & 1 & 1 \\
\end{bmatrix}
\begin{array}{ccc}
  \rightarrow R_1 \\
  \rightarrow R_2 \\
  \rightarrow R_3
\end{array}
\]

\textbf{Sol: }

$R_1 = R2$, Swapping two rows

\[
A = \begin{bmatrix}
  2 & 3 & 5 \\
  0 & 4 & 2 \\ 
  3 & 1 & 1 \\
\end{bmatrix}
\]

Apply $R_3 = 2R_3 - 3R_1$ to the above matrix 

\[
A = \begin{bmatrix}
  2 & 3 & 5 \\
  0 & 4 & 2 \\ 
  0 & -7 & -13 \\
\end{bmatrix}
\]

Apply $R_3 = 7R_2 + 4R_3$ to the above matrix 

\[
A = \begin{bmatrix}
  2 & 3 & 5 \\
  0 & 4 & 2 \\ 
  0 & 0 & -13 \\
\end{bmatrix}
\]

\textbf{H.W(2):} Make it lower triangular matrix

\newpage
\section{Systems of linear equations}
Linear system of equations can be written as \par

\bigskip

\begin{align*}
a_{11}X_{1} + a_{12}X_{2} + \cdots + a_{1n}X_{n} &= b_{1} \\
a_{21}X_{1} + a_{22}X_{2} + \cdots + a_{2n}X_{n} &= b_{2} \\
\; & \vdots \\
a_{m1}X_{1} + a_{m2}X_{2} + \cdots + a_{mn}X_{n} &= b_{m}
\end{align*}


This system can be transfer into $AX = b$

\[
A = \begin{bmatrix}
  a_{11} & a_{12} & \dots & a_{1n} \\
  a_{21} & a_{22} & \dots & a_{2n} \\
  \vdots & \vdots & \ddots & \vdots \\ 
  a_{m1} & a_{m2} & \dots & a_{mn} \\
\end{bmatrix}
\quad
X = \begin{bmatrix}
  X_{1} \\
  X_{2} \\
  \vdots \\
  x_{n}
\end{bmatrix}
\quad 
b = \begin{bmatrix}
  b_{1} \\
  b_{2} \\
  \vdots \\
  b_{n}
\end{bmatrix}
\]
\bigskip
\\To solve this system of linear equations, We can use the following methods

\begin{enumerate}
  \item \textbf{Gauss Elimination Method}
  To use this method we need to transfer the matrix $A$ into a lower or upper Triangular matrix using row operations\par
  \bigskip
  \textbf{Ex: } Solve the following system of the linear equations using Gauss method, Where\par
  \[
    \begin{aligned}
    4y + 2z = 1 \\
    2x + 3y + 5z = 0 \\
    3x + y + z = 11
    \end{aligned}
  \]

  \textbf{Sol: }
  \[
  AX = b, 
  \quad
  AX = \begin{bmatrix}
    0 & 4 & 2 \\
    2 & 3 & 5 \\
    3 & 1 & 1
  \end{bmatrix}
  \begin{bmatrix}
    x \\ y \\ z \\
  \end{bmatrix}
  b = \begin{bmatrix}
    1 \\ 0 \\ 11
  \end{bmatrix}
\]

\[
\left[
\begin{array}{ccc | c}
  0 & 4 & 2 & 1 \\
  2 & 3 & 5 & 0 \\
  3 & 1 & 1 & 11
\end{array}
\right]
\]
\[R_1 = R_2\]
\[\Downarrow\]
\[
\left[
\begin{array}{ccc | c}
  2 & 3 & 5 & 0 \\
  0 & 4 & 2 & 1 \\
  3 & 1 & 1 & 11
\end{array}
\right]
\]
\newpage
\[R_3 = 2R_3 - 3R_1\]
\[\Downarrow\]
\[
\left[
\begin{array}{ccc | c}
  2 & 3 & 5 & 0 \\
  0 & 4 & 2 & 1 \\
  0 & -7 & -13 & 22
\end{array}
\right]
\]
\[R_3 = 4R_3 - 7R_2\]
\[\Downarrow\]
\[
\left[
\begin{array}{ccc | c}
  2 & 3 & 5 & 0 \\
  0 & 4 & 2 & 1 \\
  0 & 0 & -38 & 95
\end{array}
\right]
\]

\bigskip
\[
\begin{aligned}
-38z &= 95 
    &&\Rightarrow z = -\frac{95}{38} = -2.5 \\[6pt]
4y + 2z &= 1 
    &&\Rightarrow 4y + 2(-2.5) = 1 \\
& 
    &&\Rightarrow 4y - 5 = 1 \\
&
    &&\Rightarrow 4y = 6 \Rightarrow y = \frac{3}{2} = 1.5 \\[6pt]
2x + 3y + 5z &= 0
    &&\Rightarrow 2x + 3(1.5) + 5(-2.5) = 0 \\
&
    &&\Rightarrow 2x + 4.5 - 12.5 = 0 \\
&
    &&\Rightarrow 2x - 8 = 0 \\
&
    &&\Rightarrow x = 4
\end{aligned}
\]

\item \textbf{Gauss-Jordan Elimination Method}\\
We have to transfer the system to $JX = T$
\[
J = \begin{bmatrix}
  a_{11} & 0 & 0 & \dots \\
  0 & a_{22} & 0 & \dots \\
  0 & 0 & a_{33} & \vdots \\
  0 & 0 & \dots & a_{mn} \\
\end{bmatrix}
\begin{bmatrix}
  x_1 \\ x_2 \\ \vdots \\ x_n
\end{bmatrix}
=
\begin{bmatrix}
  t_1 \\ t_2 \\ \vdots \\ t_n
\end{bmatrix}
\]

The solution will be
\[
t_1, t_2, t_3, \dots t_n
\]

\newpage
\textbf{Ex: } Solve the following system of linear equations using Gauss-Jordan
\[
\begin{aligned}
x + 2y + 3z = 9 \\
2x + 3y + z = 8 \\
3x + y + 2z = 7
\end{aligned}
\]
\textbf{Sol: }
\[
\left[
  \begin{array}{ccc | c}
    1 & 2 & 3 & 9 \\
    2 & 3 & 1 & 8 \\
    3 & 1 & 2 & 7 \\
  \end{array}
\right]
\]
\[R_2 = R_2 - 2R_1,\; R_3 = R_3 - 3R_1\]
\[\Downarrow\]
\[
\left[
  \begin{array}{ccc | c}
    1 & 2 & 3 & 9 \\
    0 & -1 & -5 & -10 \\
    0 & -5 & -7 & -20 \\
  \end{array}
\right]
\]
\[R_2 = -R_2\]
\[\Downarrow\]
\[
\left[
  \begin{array}{ccc | c}
    1 & 2 & 3 & 9 \\
    0 & 1 & 5 & 10 \\
    0 & -5 & -7 & -20 \\
  \end{array}
\right]
\]
\[R_3 = R_3 + 5R_2\]
\[\Downarrow\]
\[
\left[
  \begin{array}{ccc | c}
    1 & 2 & 3 & 9 \\
    0 & 1 & 5 & 10 \\
    0 & 0 & 18 & 30 \\
  \end{array}
\right]
\]
\[R_1 = R_1 - 2R_2\]
\[\Downarrow\]
\[
\left[
  \begin{array}{ccc | c}
    1 & 0 & -7 & -11 \\
    0 & 1 & 5 & 10 \\
    0 & 0 & 18 & 30 \\
  \end{array}
\right]
\]
\[R_1 = 18R_1 + 7R_3,\; R_2 = 18R_2 - 5R_3\]
\[\Downarrow\]
\[
\left[
  \begin{array}{ccc | c}
    18 & 0 & 0 & 12 \\
    0 & 18 & 0 & 30 \\
    0 & 0 & 18 & 30 \\
  \end{array}
\right]
\]

\[
\begin{aligned}
18x &= 12  &&\Rightarrow x = \frac{12}{18} = 0.6 \\[6pt]
18y &= 30  &&\Rightarrow y = \frac{30}{18} = 1.667 \\[6pt]
18z &= 30  &&\Rightarrow z = \frac{30}{18} = 1.667
\end{aligned}
\]

\item \textbf{General Form For Cramer's Rule}
  For a system with linear equations with $n$ variables of the form

  \begin{align*}
  a_{11}X_{1} + a_{12}X_{2} + \cdots + a_{1n}X_{n} &= b_{1} \\
  a_{21}X_{1} + a_{22}X_{2} + \cdots + a_{2n}X_{n} &= b_{2} \\
  \; & \vdots \\
  a_{m1}X_{1} + a_{m2}X_{2} + \cdots + a_{mn}X_{n} &= b_{m}
  \end{align*}

  the solution will be
  \[x_i = \frac{D_i}{D}\]
  Where 

  $D$: is the Determinant of the matrix $A$ generated from the system $AX = b$
  $D_i$: Determinant of the matrix formed by replacing the $i-th$ column of $A$ with the constant vector $b$\par
  \bigskip
  \textbf{Ex: }Solve using Cramer's rule
  \[
    \begin{aligned}
    x + 2y + z = 4 \\
    2x + y + 3z = 10 \\
    3x + 2y + 2z = 12
    \end{aligned}
  \]
  \textbf{Sol: }
  \[
  A = \begin{bmatrix}
    1 & 2 & 1 \\
    2 & 1 & 3 \\
    3 & 2 & 2 \\
  \end{bmatrix}
  \quad
  b = \begin{bmatrix}
    4 \\ 10 \\ 12 \\
  \end{bmatrix}
  \]
  \[
    D = |A| = det(A) = 1 \begin{vmatrix}
      1 & 3 \\
      2 & 2
    \end{vmatrix}
    - 2 \begin{vmatrix}
      2 & 3 \\
      3 & 2
    \end{vmatrix}
    + 1 \begin{vmatrix}
      2 & 1 \\
      3 & 2
    \end{vmatrix}
    = 7
  \]
  Now Replacing first column with the constants vector $b$
  \[
  \begin{aligned}
    D_1 &= \begin{vmatrix}
      4 & 2 & 1 \\
      10 & 1 & 3 \\
      12 & 2 & 2 \\
    \end{vmatrix}
    = 4 \begin{vmatrix}
      1 & 3 \\
      2 & 2
    \end{vmatrix}
    - 2 \begin{vmatrix}
      10 & 3 \\
      12 & 2 \\
    \end{vmatrix}
    + 1 \begin{vmatrix}
      10 & 1 \\
      12 & 2 
    \end{vmatrix} \\
    &= 4 \big(1\times2 - 3\times2 \big) - 2\big(10\times2 - 3\times12\big) + 1\big(10\times2 - 1\times12\big)\\
    &= 24 \\
  \end{aligned}
  \]
  \[
    \begin{aligned}
      D_2 &= \begin{vmatrix}
        1 & 4 & 1 \\
        2 & 10 & 3 \\
        3 & 12 & 2 \\
      \end{vmatrix}
      = 1 \begin{vmatrix}
        10 & 3 \\
        12 & 2 \\
      \end{vmatrix}
      - 4 \begin{vmatrix}
        2 & 3 \\
        3 & 2
      \end{vmatrix}
      + 1 \begin{vmatrix}
        2 & 10 \\
        3 & 12 \\
      \end{vmatrix} \\
      &= 1\big(10\times2 - 3\times12\big) - 4\big(2\times2 - 3\times3\big) +1\big(2\times12 - 10\times3\big) \\
      &= -2
    \end{aligned}
  \]
  \[
    \begin{aligned}
      D_3 &= \begin{vmatrix}
        1 & 2 & 4 \\
        2 & 1 & 10 \\
        3 & 2 & 12 \\
      \end{vmatrix}
      = 1 \begin{vmatrix}
        1 & 10 \\
        2 & 12 \\
      \end{vmatrix}
      - 2 \begin{vmatrix}
        2 & 10 \\
        3 & 12
      \end{vmatrix}
      + 4 \begin{vmatrix}
        2 & 1 \\
        3 & 2 \\
      \end{vmatrix} \\
      &= 1\big(1\times12 - 10\times2\big) - 2\big(2\times12 - 10\times3\big) +4\big(2\times2 - 1\times3\big) \\
      &= 8
    \end{aligned}
  \]
  \[
    \begin{aligned}
      x &= \frac{D_1}{D} \Rightarrow x = \frac{24}{7} \Rightarrow x \approx 3.4 \\
      y &= \frac{D_2}{D} \Rightarrow y = -\frac{2}{7} \Rightarrow y \approx 0.2 \\
      z &= \frac{D_3}{D} \Rightarrow z = \frac{8}{7} \Rightarrow z \approx 1.1 \\
    \end{aligned}
  \]

  \bigskip
  \bigskip
  \item \textbf{Inverse Matrix Method}\\
    For system of equation written in a matrix form $AX = b$ \par
    If $A \neq 0$, Then the solution of the system will be $X = A^{-1}b$, which comes from the following \par
    $AX = b \Rightarrow A^{-1}AX = A^{-1} \Rightarrow IX = A^{-1}b \Rightarrow \fbox{$X = A^{-1}b$}$ \par
    Where is a matrix with all it's diagonal elements are ones
    \begin{fleqn}
      \[ 
        I_{3\times3} = \begin{bmatrix}
          1 & 0 & 0 \\
          0 & 1 & 0 \\
          0 & 0 & 1 \\
        \end{bmatrix}
      \]
    \end{fleqn}
    \textbf{Ex(1): }solve the system using inverse method
    \[
      \begin{aligned}
        2&x + 3y = 8 \\
        &x + 4y = 7
      \end{aligned}
    \]
    \[
      A = \begin{bmatrix}
        2 & 3 \\
        1 & 4 \\
      \end{bmatrix}
      \quad
      b = \begin{bmatrix}
        8 \\ 7
      \end{bmatrix}
    \]
    \[AX = b \Rightarrow X = A^{-1}b \]
    \[ A^{-1} = \frac{adj(A)}{|A|} \]
    \[
      |A| = \begin{vmatrix}
        2 & 3 \\
        1 & 4 \\
      \end{vmatrix}
      = 5
    \]
    \[
      Adj(A) = \begin{bmatrix}
        4 & -3 \\
        -1 & 2 \\
      \end{bmatrix}
    \]
    \[
      A^{-1} = \frac{adj(A)}{|A|} = \begin{bmatrix}
        \frac{4}{5} && -\frac{3}{5} \\
        -\frac{1}{5} && \frac{2}{5} \\
      \end{bmatrix}
    \]
    \[
    \begin{aligned}
      x &= A^{-1}b \\
      \begin{bmatrix}
        x \\ y
      \end{bmatrix} &= \begin{bmatrix}
        \frac{4}{5} && -\frac{3}{5} \\
        -\frac{1}{5} && \frac{2}{5} \\
      \end{bmatrix} \times 
      \begin{bmatrix}
        8 \\ 7
      \end{bmatrix}
    \end{aligned}
    \]
    \[ x = \frac{4}{5} \times 8 - \frac{3}{5} \times 7 \Rightarrow x = \frac{32}{5} - \frac{21}{5} \Rightarrow x = \frac{11}{5} = 2.2 \]
    \[y = -\frac{1}{5} \times 8 + \frac{2}{5} \times 7 \Rightarrow y = -\frac{8}{5} + \frac{14}{5} \Rightarrow y = \frac{6}{5} \Rightarrow y = 1.2\]

    \bigskip

    \textbf{Ex(2): }
    \[
      \begin{aligned}
        x + y + z &= 6 \\
        2y + 5z &= -4 \\
        2x + 5y - z &= 27 \\
      \end{aligned}
    \]
    \textbf{Sol: }
    \[
      A = \begin{bmatrix}
        1 & 1 & 1 \\
        0 & 2 & 5 \\
        2 & 5 & -1 \\
      \end{bmatrix}
      \quad 
      b = \begin{bmatrix}
        6 \\ -4 \\ 27
      \end{bmatrix}
    \]
    \[AX = b \Rightarrow X = A^{-1}b, \text{ Where } A^{-1} = \frac{adj(A)}{|A|}\]

    \[ adj(A) = C^\top \] 
    \[
      C = \begin{vmatrix}
        1 & 1 & 1 \\
        0 & 2 & 5 \\
        2 & 5 & -1 \\
      \end{vmatrix}
    \]
    \[C_{ij} = (-1)^{i+j} M_{ij}\]
    \[
      \begin{aligned}
        C_{11} &= (-1)^{1+1} \begin{vmatrix}
          2 & 5 \\
          5 & -1
        \end{vmatrix} = 1\big[2 \times (-1) - 5 \times 5\big] = -27 \\
        %
        C_{12} &= (-1)^{1+2} \begin{vmatrix}
          0 & 5 \\
          2 & -1
        \end{vmatrix} = -1\big[0 \times (-1) - 5 \times 2\big] = 10\\
        % 
        C_{13} &= (-1)^{1+3} \begin{vmatrix}
          0 & 2 \\
          2 & 5
        \end{vmatrix} = 1\big[0 \times 5 - 2 \times 2\big] = -4\\
        % 
        C_{21} &= (-1)^{2+1} \begin{vmatrix}
          1 & 1 \\
          5 & -1
        \end{vmatrix} = -1\big[1 \times (-1) - 1 \times 5\big] = 6\\
        % 
        C_{22} &= (-1)^{2+2} \begin{vmatrix}
          1 & 1 \\
          2 & -1
        \end{vmatrix} = 1\big[1 \times (-1) - 1 \times 2\big]= -3\\
        % 
        C_{23} &= (-1)^{2+3} \begin{vmatrix}
          1 & 1 \\
          2 & 5
        \end{vmatrix} = -1\big[1 \times 5 - 1 \times 2\big] = -3\\
        % 
        C_{31} &= (-1)^{3+1} \begin{vmatrix}
          1 & 1 \\
          2 & 5
        \end{vmatrix} = 1\big[1 \times 5 - 1 \times 2\big] = 3\\
        % 
        C_{32} &= (-1)^{3+2} \begin{vmatrix}
          1 & 1 \\
          0 & 5
        \end{vmatrix} = -1\big[1 \times 5 - 1 \times 0\big] = -5\\
        % 
        C_{33} &= (-1)^{3+3} \begin{vmatrix}
          1 & 1 \\
          0 & 2
        \end{vmatrix} = 1\big[1 \times 2 - 1 \times 0 \big]= 2\\
      \end{aligned}
    \]
    \[
      C = \begin{bmatrix}
          -27 & 10 & -4 \\
          6 & -3 & -3 \\
          3 & -5 & 2
        \end{bmatrix} 
    \]

    \[
    \operatorname{adj}(A) = C^\top = \begin{bmatrix}
    -27 & 6 & 3 \\
    10 & -3 & -5 \\
    -4 & -3 & 2
    \end{bmatrix}
    \]
    Finding the cofactor
    \[
      \begin{aligned}
        |A| &= 1 \begin{vmatrix}
          2 & 5 \\
          5 & -1 \\
        \end{vmatrix}
        - 1 \begin{vmatrix}
          0 & 5 \\
          2 & -1 \\
        \end{vmatrix}
        + 1 \begin{vmatrix}
          0 & 2 \\
          2 & 5
        \end{vmatrix} \\
        |A| &= 1(2\times-1 - 5\times5) - 1(0\times-1 - 5\times2)\\ 
            &+ 1(0\times5 - 2\times2) \\
        |A| &= -21
      \end{aligned}
    \]

    \[
      A^{-1} = \frac{\operatorname{adj}(A)}{|A|} = \begin{bmatrix}
        \frac{-27}{-21} & \frac{6}{-21} & \frac{3}{-21} \\[6pt]
        \frac{10}{-21} & \frac{-3}{-21} & \frac{-5}{-21} \\[6pt]
        \frac{-4}{-21} & \frac{-3}{-21} & \frac{2}{-21}
      \end{bmatrix}
      \Rightarrow
      A^{-1} = \begin{bmatrix}
        \frac{9}{7} & -\frac{2}{7} & -\frac{1}{7} \\[6pt]
        -\frac{10}{21} & \frac{1}{7} & \frac{5}{21} \\[6pt]
        \frac{4}{21} & \frac{1}{7} & -\frac{2}{21}
      \end{bmatrix}
    \]

    \[
      \begin{aligned}
        X &= A^{-1}b \\
        \begin{bmatrix}
          x \\[6pt] y \\[6pt] z
        \end{bmatrix}
        &= \begin{bmatrix}
          \frac{9}{7} & -\frac{2}{7} & -\frac{1}{7} \\[6pt]
          -\frac{10}{21} & \frac{1}{7} & \frac{5}{21} \\[6pt]
          \frac{4}{21} & \frac{1}{7} & -\frac{2}{21}
        \end{bmatrix}
        \times
        \begin{bmatrix}
          6 \\[6pt] -4 \\[6pt] 27
        \end{bmatrix}
      \end{aligned}
    \]
    \[ x = \frac{9}{7}\times6 + \frac{2}{7}\times4 - \frac{1}{7}\times27 \Rightarrow x = 5 \]
    \[ y = -\frac{10}{21}\times6 - \frac{1}{7}\times 4 + \frac{5}{21}\times27 \Rightarrow y = 3 \]
    \[ z = \frac{4}{21}\times6 - \frac{1}{7}\times4 -\frac{2}{21}\times27 \Rightarrow z = -2 \]

    \bigskip
    \item \textbf{Lower/Upper Decomposition Method} \par
    In this method we will write $A = LU$, Where \par
    $L$: Lower triangular matrix with ($1$)s on the main diagonal\par
    $U$: Upper triangular Matrix \par
    Then \par
    \[
    L = \begin{bmatrix}
      1 & 0 & 0 \\
      l_{21} & 1 & 0 \\
      l_{31} & l_{32} & 1 \\
    \end{bmatrix}
    \quad
    U = \begin{bmatrix}
      u_{11} & u_{12} & u_{13} \\
      0 & u_{22} & u_{23} \\
      0 & 0 & u_{33} \\
    \end{bmatrix}
    \]
    Which will produce the following simplification
    \[
    \begin{aligned}
      A &= LU \\
      LU &=\begin{bmatrix}
        1 \times u_{11} + 0 \times 0 + 0 \times 0 & 1 \times u_{12} + 0 \times u_{22} + 0 \times 0 & 1 \times u_{13} + 0 \times u_{23} + 0 \times u_{33} \\[5pt]
        l_{21}u_{11} + 1\times 0 + 0 \times 0 & l_{21}u_{12} + 1 \times u_{22} + 0 \times 0 & l_{21}u_{13} + 1 \times u_{23} + 0 \times u_{33} \\[5pt]
        l_{31}u_{11} + l_{32} \times 0 + 1 \times 0 & l_{31}u_{12} + l_{32}u_{22} + 1 \times 0 &  l_{31}u_{13} + l_{32}u_{23} + 1 \times u_{33} \\
      \end{bmatrix} \\
      &\text{Simplifies this}\\
      A &= \begin{bmatrix}
        a_{11} & a_{12} & a_{13} \\
        a_{21} & a_{22} & a_{23} \\
        a_{31} & a_{32} & a_{33} \\
      \end{bmatrix} = LU \\
      A &= \begin{bmatrix}
        u_{11} & u_{12} & u_{13} \\
        l_{21}u_{11} & l_{21}u_{12} + u_{22} &  l_{21}u_{13} + u_{23} \\
        l_{31}u_{11} & l_{31}u_{12} + l_{32}u_{22} &  l_{31}u_{13} + l_{32}u_{23} + u_{33} \\
      \end{bmatrix} \\
    \end{aligned}\\
    \]

    \newpage
    This will produce this equations
    \[u_{11} = a_{11}\]
    \[u_{12} = a_{12}\]
    \[u_{13} = a_{13}\]
    \[l_{21}u_{11} = a_{21}\]
    \[l_{21}u_{12}+u_{22} = a_{22}\]
    \[l_{21}u_{13} + u_{23} = a_{23}\]
    \[l_{31}u_{11} = a_{31}\]
    \[l_{31}u_{12} + l_{32}u_{22} = a_{32}\]
    \[l_{31}u_{13} + l_{32}u_{23} + u_{33} = a_{33}\]

    \textbf{Ex: }
    \[
      \begin{aligned}
        2x + 3y + z &= 1\\
        4x + 7y + 5z &= 2 \\
        6x + 18y + 19z &= 3 \\
      \end{aligned}
    \]
    \textbf{Sol: }
    \[
      A = \begin{bmatrix}
        2 & 3 & 1 \\
        4 & 7 & 5 \\
        6 & 18 & 19 \\
      \end{bmatrix}
      = \begin{bmatrix}
        1 \\ 2 \\ 3
      \end{bmatrix}
    \]
    \[
      \begin{aligned}
        u_{11} &= a_{11} = 2 \\
        u_{12} &= a_{12} = 3 \\
        u_{13} &= a_{13} = 1 \\
        \\
        l_{21}u_{11} &= a_{21} \\ 
        l_{21} \times 2 &= 4 \\
        l_{21}&= 2 \\
        \\
        l_{21}u_{12}+u_{22} &= a_{22} \\
        2 \times 3 + u_{22} &= 7 \\
        6 + u_{22} &= 7 \\
        u_{22} &= 7 - 6 \\
        u_{22} &= 1 \\
        \\
        l_{21}u_{13} + u_{23} &= a_{23}\\
        2 \times 1 + u_{23} &= 5 \\
        2 + u_{23} &= 5\\
        u_{23} &= 5 - 2\\
        u_{23} &= 3\\
      \end{aligned}
    \]
    \[
      \begin{aligned}
        l_{31}u_{11} &= a_{31}\\
        l_{31}2 &= 6\\
        l_{31} &= 3 \\
        \\
        l_{31}u_{12} + l_{32}u_{22} &= a_{32} \\
        3\times 3 + l_{32} 1 &= 18 \\
        9 + l_{32} &= 18 \\
        l_{32} &= 18 - 9 \\
        l_{32} &= 9 \\
        \\
        l_{31}u_{13} + l_{32}u_{23} + u_{33} &= a_{33} \\
        3 \times 1 + 9 \times 3 + u_{33} &= 19 \\
        3 + 27 + u_{33} &= 19 \\
        30 + u_{33} &= 19 \\
        u_{33} &= 19 - 30 \\
        u_{33} &= -11\\
      \end{aligned}
    \]
    \[
      L = \begin{bmatrix}
        1 & 0 & 0 \\
        2 & 1 & 0 \\
        3 & 9 & 1
      \end{bmatrix}, 
      \quad
      U = \begin{bmatrix}
        2 & 3 & 1 \\
        0 & 1 & 3 \\
        0 & 0 & -11
      \end{bmatrix}
    \]
    \[
      \begin{aligned}
        Ly &= b \\
        \begin{bmatrix}
          1 & 0 & 0 \\
          2 & 1 & 0 \\
          3 & 9 & 1
      \end{bmatrix}
      \begin{bmatrix}
        y_1 \\ y_2 \\ y_3 
      \end{bmatrix}
      &= \begin{bmatrix}
        1 \\ 2 \\ 3
      \end{bmatrix}
      \end{aligned}
    \]
    \[
      \begin{aligned}
        1 \times y_1 + 0 \times y_2 + 0 \times y_3 &= 1 \\
        \\
        2 \times y_1 + 1 \times y_2 + 0 \times y_3 &= 2 \\
        2 \times 1 + y_2 &= 2 \\
        2 + y_2 &= 2 \\
        y2 &= 2 - 2 \\
        y_2 &= 0 \\
        \\
        3 \times y_1 + 9 \times y_2 + 1 \times y_3 &= 3 \\
        3 \times 1 + 9 \times 0 + y_3 &= 3 \\
        3 + 0 + y_3 &= 3 \\
        y_3 &= 3 - 3 \\
        y_3 &= 0 \\
      \end{aligned}
    \]
    \newpage
    \[
      y = \begin{bmatrix}
        1 \\ 0 \\ 0
      \end{bmatrix}
    \]
    \[\begin{aligned}
      ux &= y \\
      \begin{bmatrix}
        2 & 3 & 1 \\
        0 & 1 & 3 \\
        0 & 0 & -11
      \end{bmatrix}
    \begin{bmatrix}
      x \\ y \\ z
    \end{bmatrix}
    &= \begin{bmatrix}
        1 \\ 0 \\ 0
      \end{bmatrix}
    \end{aligned}
  \]
  From the third row
  \[
    \begin{aligned}
      0 \times x + 0 \times y -11 \times z &= 0\\
      -11z &= 0 \\
      z &= 0
    \end{aligned}
  \]
  The second row
  \[
    \begin{aligned}
    0 \times x + 1 \times y + 3 \times z &= 0 \\
    0 + y + 3(0) = 0 \\
    y &= 0
    \end{aligned}
  \]
  The first row
  \[
    \begin{aligned}
      2 \times x + 3 \times y + 1 \times z &= 1\\
      2x + 3 \times 0 + 1 \times 0 &= 1 \\
      2x &= 1 \\
      x &= \frac{1}{2}\\
    \end{aligned}
  \]
\end{enumerate}

\end{document}